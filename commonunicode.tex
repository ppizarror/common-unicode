\documentclass{article}
\usepackage[T1]{fontenc}
\usepackage{hyperref}
\hypersetup{colorlinks}
\usepackage{commonunicode}

\title{The commonunicode package}
\author{Pablo Pizarro R. @ \href{https://ppizarror.com/}{ppizarror.com}}

\date{\csname ver@commonunicode.sty\endcsname}

\begin{document}
	
\maketitle
\tableofcontents

\section{Introduction}

The common-unicode package allows you to insert unicode characters into any latex document without the need to use complex commands, 𝑠𝑢𝑐𝘩 𝑎𝑠 𝑡𝘩𝑖𝑠 𝑜𝑛𝑒. \\

\noindent To use this package:

\textbackslash\texttt{usepackage\{commonunicode\}} \\

As there is thousands of different characters, you can provide new ones! Simply create a new issue on the \href{https://github.com/ppizarror/common-unicode}{Github common-unicode repo}.

\section{List of added unicodes}
\begin{itemize}
	\item \href{https://www.compart.com/en/unicode/U+000B}{U+000B}: 
	\item \href{https://www.compart.com/en/unicode/U+00A0}{U+00A0}:  
	\item \href{https://www.compart.com/en/unicode/U+00A1}{U+00A1}: ¡
	\item \href{https://www.compart.com/en/unicode/U+00A2}{U+00A2}: ¢
	\item \href{https://www.compart.com/en/unicode/U+00A3}{U+00A3}: £
	\item \href{https://www.compart.com/en/unicode/U+00A4}{U+00A4}: ¤
	\item \href{https://www.compart.com/en/unicode/U+00A5}{U+00A5}: ¥
	\item \href{https://www.compart.com/en/unicode/U+00A6}{U+00A6}: ¦
	\item \href{https://www.compart.com/en/unicode/U+00A7}{U+00A7}: §
	\item \href{https://www.compart.com/en/unicode/U+00A8}{U+00A8}: ¨
	\item \href{https://www.compart.com/en/unicode/U+00A9}{U+00A9}: ©
	\item \href{https://www.compart.com/en/unicode/U+00AA}{U+00AA}: ª
	\item \href{https://www.compart.com/en/unicode/U+00AB}{U+00AB}: «
	\item \href{https://www.compart.com/en/unicode/U+00AC}{U+00AC}: ¬
	\item \href{https://www.compart.com/en/unicode/U+00AE}{U+00AE}: ®
	\item \href{https://www.compart.com/en/unicode/U+00AF}{U+00AF}: ¯
	\item \href{https://www.compart.com/en/unicode/U+00B0}{U+00B0}: °
	\item \href{https://www.compart.com/en/unicode/U+00B1}{U+00B1}: ±
	\item \href{https://www.compart.com/en/unicode/U+00B2}{U+00B2}: ²
	\item \href{https://www.compart.com/en/unicode/U+00B3}{U+00B3}: ³
	\item \href{https://www.compart.com/en/unicode/U+00B5}{U+00B5}: µ
	\item \href{https://www.compart.com/en/unicode/U+00B6}{U+00B6}: ¶
	\item \href{https://www.compart.com/en/unicode/U+00B7}{U+00B7}: ·
	\item \href{https://www.compart.com/en/unicode/U+00B9}{U+00B9}: ¹
	\item \href{https://www.compart.com/en/unicode/U+00BA}{U+00BA}: º
	\item \href{https://www.compart.com/en/unicode/U+00BB}{U+00BB}: »
	\item \href{https://www.compart.com/en/unicode/U+00BC}{U+00BC}: ¼
	\item \href{https://www.compart.com/en/unicode/U+00BD}{U+00BD}: ½
	\item \href{https://www.compart.com/en/unicode/U+00BE}{U+00BE}: ¾
	\item \href{https://www.compart.com/en/unicode/U+00BF}{U+00BF}: ¿
	\item \href{https://www.compart.com/en/unicode/U+00D7}{U+00D7}: ×
	\item \href{https://www.compart.com/en/unicode/U+00F7}{U+00F7}: ÷
	\item \href{https://www.compart.com/en/unicode/U+0131}{U+0131}: ı
	\item \href{https://www.compart.com/en/unicode/U+02102}{U+02102}: ℂ
	\item \href{https://www.compart.com/en/unicode/U+0210D}{U+0210D}: ℍ
	\item \href{https://www.compart.com/en/unicode/U+02115}{U+02115}: ℕ
	\item \href{https://www.compart.com/en/unicode/U+02119}{U+02119}: ℙ
	\item \href{https://www.compart.com/en/unicode/U+0211A}{U+0211A}: ℚ
	\item \href{https://www.compart.com/en/unicode/U+0211D}{U+0211D}: ℝ
	\item \href{https://www.compart.com/en/unicode/U+02124}{U+02124}: ℤ
	\item \href{https://www.compart.com/en/unicode/U+0237}{U+0237}: ȷ
	\item \href{https://www.compart.com/en/unicode/U+02B0}{U+02B0}: ʰ
	\item \href{https://www.compart.com/en/unicode/U+02B2}{U+02B2}: ʲ
	\item \href{https://www.compart.com/en/unicode/U+02B3}{U+02B3}: ʳ
	\item \href{https://www.compart.com/en/unicode/U+02B7}{U+02B7}: ʷ
	\item \href{https://www.compart.com/en/unicode/U+02B8}{U+02B8}: ʸ
	\item \href{https://www.compart.com/en/unicode/U+02E1}{U+02E1}: ˡ
	\item \href{https://www.compart.com/en/unicode/U+02E2}{U+02E2}: ˢ
	\item \href{https://www.compart.com/en/unicode/U+02E3}{U+02E3}: ˣ
	\item \href{https://www.compart.com/en/unicode/U+0302}{U+0302}: ̂
	\item \href{https://www.compart.com/en/unicode/U+0308}{U+0308}: ̈
	\item \href{https://www.compart.com/en/unicode/U+0332}{U+0332}: ̲
	\item \href{https://www.compart.com/en/unicode/U+0391}{U+0391}: Α
	\item \href{https://www.compart.com/en/unicode/U+0392}{U+0392}: Β
	\item \href{https://www.compart.com/en/unicode/U+0393}{U+0393}: Γ
	\item \href{https://www.compart.com/en/unicode/U+0394}{U+0394}: Δ
	\item \href{https://www.compart.com/en/unicode/U+0395}{U+0395}: Ε
	\item \href{https://www.compart.com/en/unicode/U+0396}{U+0396}: Ζ
	\item \href{https://www.compart.com/en/unicode/U+0397}{U+0397}: Η
	\item \href{https://www.compart.com/en/unicode/U+0398}{U+0398}: Θ
	\item \href{https://www.compart.com/en/unicode/U+0399}{U+0399}: Ι
	\item \href{https://www.compart.com/en/unicode/U+039A}{U+039A}: Κ
	\item \href{https://www.compart.com/en/unicode/U+039B}{U+039B}: Λ
	\item \href{https://www.compart.com/en/unicode/U+039C}{U+039C}: Μ
	\item \href{https://www.compart.com/en/unicode/U+039D}{U+039D}: Ν
	\item \href{https://www.compart.com/en/unicode/U+039E}{U+039E}: Ξ
	\item \href{https://www.compart.com/en/unicode/U+039F}{U+039F}: Ο
	\item \href{https://www.compart.com/en/unicode/U+03A0}{U+03A0}: Π
	\item \href{https://www.compart.com/en/unicode/U+03A1}{U+03A1}: Ρ
	\item \href{https://www.compart.com/en/unicode/U+03A3}{U+03A3}: Σ
	\item \href{https://www.compart.com/en/unicode/U+03A4}{U+03A4}: Τ
	\item \href{https://www.compart.com/en/unicode/U+03A5}{U+03A5}: Υ
	\item \href{https://www.compart.com/en/unicode/U+03A6}{U+03A6}: Φ
	\item \href{https://www.compart.com/en/unicode/U+03A7}{U+03A7}: Χ
	\item \href{https://www.compart.com/en/unicode/U+03A8}{U+03A8}: Ψ
	\item \href{https://www.compart.com/en/unicode/U+03A9}{U+03A9}: Ω
	\item \href{https://www.compart.com/en/unicode/U+03B1}{U+03B1}: α
	\item \href{https://www.compart.com/en/unicode/U+03B2}{U+03B2}: β
	\item \href{https://www.compart.com/en/unicode/U+03B3}{U+03B3}: γ
	\item \href{https://www.compart.com/en/unicode/U+03B4}{U+03B4}: δ
	\item \href{https://www.compart.com/en/unicode/U+03B5}{U+03B5}: ε
	\item \href{https://www.compart.com/en/unicode/U+03B6}{U+03B6}: ζ
	\item \href{https://www.compart.com/en/unicode/U+03B7}{U+03B7}: η
	\item \href{https://www.compart.com/en/unicode/U+03B8}{U+03B8}: θ
	\item \href{https://www.compart.com/en/unicode/U+03B9}{U+03B9}: ι
	\item \href{https://www.compart.com/en/unicode/U+03BA}{U+03BA}: κ
	\item \href{https://www.compart.com/en/unicode/U+03BB}{U+03BB}: λ
	\item \href{https://www.compart.com/en/unicode/U+03BC}{U+03BC}: μ
	\item \href{https://www.compart.com/en/unicode/U+03BD}{U+03BD}: ν
	\item \href{https://www.compart.com/en/unicode/U+03BE}{U+03BE}: ξ
	\item \href{https://www.compart.com/en/unicode/U+03BF}{U+03BF}: ο
	\item \href{https://www.compart.com/en/unicode/U+03C0}{U+03C0}: π
	\item \href{https://www.compart.com/en/unicode/U+03C1}{U+03C1}: ρ
	\item \href{https://www.compart.com/en/unicode/U+03C2}{U+03C2}: ς
	\item \href{https://www.compart.com/en/unicode/U+03C3}{U+03C3}: σ
	\item \href{https://www.compart.com/en/unicode/U+03C4}{U+03C4}: τ
	\item \href{https://www.compart.com/en/unicode/U+03C5}{U+03C5}: υ
	\item \href{https://www.compart.com/en/unicode/U+03C6}{U+03C6}: φ
	\item \href{https://www.compart.com/en/unicode/U+03C7}{U+03C7}: χ
	\item \href{https://www.compart.com/en/unicode/U+03C8}{U+03C8}: ψ
	\item \href{https://www.compart.com/en/unicode/U+03C9}{U+03C9}: ω
	\item \href{https://www.compart.com/en/unicode/U+03D0}{U+03D0}: ϐ
	\item \href{https://www.compart.com/en/unicode/U+03D1}{U+03D1}: ϑ
	\item \href{https://www.compart.com/en/unicode/U+03D5}{U+03D5}: ϕ
	\item \href{https://www.compart.com/en/unicode/U+03D6}{U+03D6}: ϖ
	\item \href{https://www.compart.com/en/unicode/U+03D8}{U+03D8}: Ϙ
	\item \href{https://www.compart.com/en/unicode/U+03D9}{U+03D9}: ϙ
	\item \href{https://www.compart.com/en/unicode/U+03DA}{U+03DA}: Ϛ
	\item \href{https://www.compart.com/en/unicode/U+03DB}{U+03DB}: ϛ
	\item \href{https://www.compart.com/en/unicode/U+03DC}{U+03DC}: Ϝ
	\item \href{https://www.compart.com/en/unicode/U+03DD}{U+03DD}: ϝ
	\item \href{https://www.compart.com/en/unicode/U+03DE}{U+03DE}: Ϟ
	\item \href{https://www.compart.com/en/unicode/U+03DF}{U+03DF}: ϟ
	\item \href{https://www.compart.com/en/unicode/U+03E0}{U+03E0}: Ϡ
	\item \href{https://www.compart.com/en/unicode/U+03E1}{U+03E1}: ϡ
	\item \href{https://www.compart.com/en/unicode/U+03F0}{U+03F0}: ϰ
	\item \href{https://www.compart.com/en/unicode/U+03F1}{U+03F1}: ϱ
	\item \href{https://www.compart.com/en/unicode/U+03F5}{U+03F5}: ϵ
	\item \href{https://www.compart.com/en/unicode/U+03F6}{U+03F6}: ϶
	\item \href{https://www.compart.com/en/unicode/U+041F}{U+041F}: П
	\item \href{https://www.compart.com/en/unicode/U+0432}{U+0432}: в
	\item \href{https://www.compart.com/en/unicode/U+0435}{U+0435}: е
	\item \href{https://www.compart.com/en/unicode/U+0438}{U+0438}: и
	\item \href{https://www.compart.com/en/unicode/U+043C}{U+043C}: м
	\item \href{https://www.compart.com/en/unicode/U+0440}{U+0440}: р
	\item \href{https://www.compart.com/en/unicode/U+0442}{U+0442}: т
	\item \href{https://www.compart.com/en/unicode/U+0BA8}{U+0BA8}: ந
	\item \href{https://www.compart.com/en/unicode/U+0BBF}{U+0BBF}: ி
	\item \href{https://www.compart.com/en/unicode/U+1100}{U+1100}: ᄀ
	\item \href{https://www.compart.com/en/unicode/U+11F9}{U+11F9}: ᇹ
	\item \href{https://www.compart.com/en/unicode/U+1D2C}{U+1D2C}: ᴬ
	\item \href{https://www.compart.com/en/unicode/U+1D2E}{U+1D2E}: ᴮ
	\item \href{https://www.compart.com/en/unicode/U+1D30}{U+1D30}: ᴰ
	\item \href{https://www.compart.com/en/unicode/U+1D31}{U+1D31}: ᴱ
	\item \href{https://www.compart.com/en/unicode/U+1D33}{U+1D33}: ᴳ
	\item \href{https://www.compart.com/en/unicode/U+1D34}{U+1D34}: ᴴ
	\item \href{https://www.compart.com/en/unicode/U+1D35}{U+1D35}: ᴵ
	\item \href{https://www.compart.com/en/unicode/U+1D36}{U+1D36}: ᴶ
	\item \href{https://www.compart.com/en/unicode/U+1D37}{U+1D37}: ᴷ
	\item \href{https://www.compart.com/en/unicode/U+1D38}{U+1D38}: ᴸ
	\item \href{https://www.compart.com/en/unicode/U+1D39}{U+1D39}: ᴹ
	\item \href{https://www.compart.com/en/unicode/U+1D3A}{U+1D3A}: ᴺ
	\item \href{https://www.compart.com/en/unicode/U+1D3C}{U+1D3C}: ᴼ
	\item \href{https://www.compart.com/en/unicode/U+1D3E}{U+1D3E}: ᴾ
	\item \href{https://www.compart.com/en/unicode/U+1D3F}{U+1D3F}: ᴿ
	\item \href{https://www.compart.com/en/unicode/U+1D40}{U+1D40}: ᵀ
	\item \href{https://www.compart.com/en/unicode/U+1D400}{U+1D400}: 𝐀
	\item \href{https://www.compart.com/en/unicode/U+1D401}{U+1D401}: 𝐁
	\item \href{https://www.compart.com/en/unicode/U+1D402}{U+1D402}: 𝐂
	\item \href{https://www.compart.com/en/unicode/U+1D403}{U+1D403}: 𝐃
	\item \href{https://www.compart.com/en/unicode/U+1D404}{U+1D404}: 𝐄
	\item \href{https://www.compart.com/en/unicode/U+1D405}{U+1D405}: 𝐅
	\item \href{https://www.compart.com/en/unicode/U+1D406}{U+1D406}: 𝐆
	\item \href{https://www.compart.com/en/unicode/U+1D407}{U+1D407}: 𝐇
	\item \href{https://www.compart.com/en/unicode/U+1D408}{U+1D408}: 𝐈
	\item \href{https://www.compart.com/en/unicode/U+1D409}{U+1D409}: 𝐉
	\item \href{https://www.compart.com/en/unicode/U+1D40A}{U+1D40A}: 𝐊
	\item \href{https://www.compart.com/en/unicode/U+1D40B}{U+1D40B}: 𝐋
	\item \href{https://www.compart.com/en/unicode/U+1D40C}{U+1D40C}: 𝐌
	\item \href{https://www.compart.com/en/unicode/U+1D40D}{U+1D40D}: 𝐍
	\item \href{https://www.compart.com/en/unicode/U+1D40E}{U+1D40E}: 𝐎
	\item \href{https://www.compart.com/en/unicode/U+1D40F}{U+1D40F}: 𝐏
	\item \href{https://www.compart.com/en/unicode/U+1D41}{U+1D41}: ᵁ
	\item \href{https://www.compart.com/en/unicode/U+1D410}{U+1D410}: 𝐐
	\item \href{https://www.compart.com/en/unicode/U+1D411}{U+1D411}: 𝐑
	\item \href{https://www.compart.com/en/unicode/U+1D412}{U+1D412}: 𝐒
	\item \href{https://www.compart.com/en/unicode/U+1D413}{U+1D413}: 𝐓
	\item \href{https://www.compart.com/en/unicode/U+1D414}{U+1D414}: 𝐔
	\item \href{https://www.compart.com/en/unicode/U+1D415}{U+1D415}: 𝐕
	\item \href{https://www.compart.com/en/unicode/U+1D416}{U+1D416}: 𝐖
	\item \href{https://www.compart.com/en/unicode/U+1D417}{U+1D417}: 𝐗
	\item \href{https://www.compart.com/en/unicode/U+1D418}{U+1D418}: 𝐘
	\item \href{https://www.compart.com/en/unicode/U+1D419}{U+1D419}: 𝐙
	\item \href{https://www.compart.com/en/unicode/U+1D41A}{U+1D41A}: 𝐚
	\item \href{https://www.compart.com/en/unicode/U+1D41B}{U+1D41B}: 𝐛
	\item \href{https://www.compart.com/en/unicode/U+1D41C}{U+1D41C}: 𝐜
	\item \href{https://www.compart.com/en/unicode/U+1D41D}{U+1D41D}: 𝐝
	\item \href{https://www.compart.com/en/unicode/U+1D41E}{U+1D41E}: 𝐞
	\item \href{https://www.compart.com/en/unicode/U+1D41F}{U+1D41F}: 𝐟
	\item \href{https://www.compart.com/en/unicode/U+1D42}{U+1D42}: ᵂ
	\item \href{https://www.compart.com/en/unicode/U+1D420}{U+1D420}: 𝐠
	\item \href{https://www.compart.com/en/unicode/U+1D421}{U+1D421}: 𝐡
	\item \href{https://www.compart.com/en/unicode/U+1D422}{U+1D422}: 𝐢
	\item \href{https://www.compart.com/en/unicode/U+1D423}{U+1D423}: 𝐣
	\item \href{https://www.compart.com/en/unicode/U+1D424}{U+1D424}: 𝐤
	\item \href{https://www.compart.com/en/unicode/U+1D425}{U+1D425}: 𝐥
	\item \href{https://www.compart.com/en/unicode/U+1D426}{U+1D426}: 𝐦
	\item \href{https://www.compart.com/en/unicode/U+1D427}{U+1D427}: 𝐧
	\item \href{https://www.compart.com/en/unicode/U+1D428}{U+1D428}: 𝐨
	\item \href{https://www.compart.com/en/unicode/U+1D429}{U+1D429}: 𝐩
	\item \href{https://www.compart.com/en/unicode/U+1D42A}{U+1D42A}: 𝐪
	\item \href{https://www.compart.com/en/unicode/U+1D42B}{U+1D42B}: 𝐫
	\item \href{https://www.compart.com/en/unicode/U+1D42C}{U+1D42C}: 𝐬
	\item \href{https://www.compart.com/en/unicode/U+1D42D}{U+1D42D}: 𝐭
	\item \href{https://www.compart.com/en/unicode/U+1D42E}{U+1D42E}: 𝐮
	\item \href{https://www.compart.com/en/unicode/U+1D42F}{U+1D42F}: 𝐯
	\item \href{https://www.compart.com/en/unicode/U+1D43}{U+1D43}: ᵃ
	\item \href{https://www.compart.com/en/unicode/U+1D430}{U+1D430}: 𝐰
	\item \href{https://www.compart.com/en/unicode/U+1D431}{U+1D431}: 𝐱
	\item \href{https://www.compart.com/en/unicode/U+1D432}{U+1D432}: 𝐲
	\item \href{https://www.compart.com/en/unicode/U+1D433}{U+1D433}: 𝐳
	\item \href{https://www.compart.com/en/unicode/U+1D434}{U+1D434}: 𝐴
	\item \href{https://www.compart.com/en/unicode/U+1D435}{U+1D435}: 𝐵
	\item \href{https://www.compart.com/en/unicode/U+1D436}{U+1D436}: 𝐶
	\item \href{https://www.compart.com/en/unicode/U+1D437}{U+1D437}: 𝐷
	\item \href{https://www.compart.com/en/unicode/U+1D438}{U+1D438}: 𝐸
	\item \href{https://www.compart.com/en/unicode/U+1D439}{U+1D439}: 𝐹
	\item \href{https://www.compart.com/en/unicode/U+1D43A}{U+1D43A}: 𝐺
	\item \href{https://www.compart.com/en/unicode/U+1D43B}{U+1D43B}: 𝐻
	\item \href{https://www.compart.com/en/unicode/U+1D43C}{U+1D43C}: 𝐼
	\item \href{https://www.compart.com/en/unicode/U+1D43D}{U+1D43D}: 𝐽
	\item \href{https://www.compart.com/en/unicode/U+1D43E}{U+1D43E}: 𝐾
	\item \href{https://www.compart.com/en/unicode/U+1D43F}{U+1D43F}: 𝐿
	\item \href{https://www.compart.com/en/unicode/U+1D440}{U+1D440}: 𝑀
	\item \href{https://www.compart.com/en/unicode/U+1D441}{U+1D441}: 𝑁
	\item \href{https://www.compart.com/en/unicode/U+1D442}{U+1D442}: 𝑂
	\item \href{https://www.compart.com/en/unicode/U+1D443}{U+1D443}: 𝑃
	\item \href{https://www.compart.com/en/unicode/U+1D444}{U+1D444}: 𝑄
	\item \href{https://www.compart.com/en/unicode/U+1D445}{U+1D445}: 𝑅
	\item \href{https://www.compart.com/en/unicode/U+1D446}{U+1D446}: 𝑆
	\item \href{https://www.compart.com/en/unicode/U+1D447}{U+1D447}: 𝑇
	\item \href{https://www.compart.com/en/unicode/U+1D448}{U+1D448}: 𝑈
	\item \href{https://www.compart.com/en/unicode/U+1D449}{U+1D449}: 𝑉
	\item \href{https://www.compart.com/en/unicode/U+1D44A}{U+1D44A}: 𝑊
	\item \href{https://www.compart.com/en/unicode/U+1D44B}{U+1D44B}: 𝑋
	\item \href{https://www.compart.com/en/unicode/U+1D44C}{U+1D44C}: 𝑌
	\item \href{https://www.compart.com/en/unicode/U+1D44D}{U+1D44D}: 𝑍
	\item \href{https://www.compart.com/en/unicode/U+1D44E}{U+1D44E}: 𝑎
	\item \href{https://www.compart.com/en/unicode/U+1D44F}{U+1D44F}: 𝑏
	\item \href{https://www.compart.com/en/unicode/U+1D450}{U+1D450}: 𝑐
	\item \href{https://www.compart.com/en/unicode/U+1D451}{U+1D451}: 𝑑
	\item \href{https://www.compart.com/en/unicode/U+1D452}{U+1D452}: 𝑒
	\item \href{https://www.compart.com/en/unicode/U+1D453}{U+1D453}: 𝑓
	\item \href{https://www.compart.com/en/unicode/U+1D454}{U+1D454}: 𝑔
	\item \href{https://www.compart.com/en/unicode/U+1D456}{U+1D456}: 𝑖
	\item \href{https://www.compart.com/en/unicode/U+1D457}{U+1D457}: 𝑗
	\item \href{https://www.compart.com/en/unicode/U+1D458}{U+1D458}: 𝑘
	\item \href{https://www.compart.com/en/unicode/U+1D459}{U+1D459}: 𝑙
	\item \href{https://www.compart.com/en/unicode/U+1D45A}{U+1D45A}: 𝑚
	\item \href{https://www.compart.com/en/unicode/U+1D45B}{U+1D45B}: 𝑛
	\item \href{https://www.compart.com/en/unicode/U+1D45C}{U+1D45C}: 𝑜
	\item \href{https://www.compart.com/en/unicode/U+1D45D}{U+1D45D}: 𝑝
	\item \href{https://www.compart.com/en/unicode/U+1D45E}{U+1D45E}: 𝑞
	\item \href{https://www.compart.com/en/unicode/U+1D45F}{U+1D45F}: 𝑟
	\item \href{https://www.compart.com/en/unicode/U+1D460}{U+1D460}: 𝑠
	\item \href{https://www.compart.com/en/unicode/U+1D461}{U+1D461}: 𝑡
	\item \href{https://www.compart.com/en/unicode/U+1D462}{U+1D462}: 𝑢
	\item \href{https://www.compart.com/en/unicode/U+1D463}{U+1D463}: 𝑣
	\item \href{https://www.compart.com/en/unicode/U+1D464}{U+1D464}: 𝑤
	\item \href{https://www.compart.com/en/unicode/U+1D465}{U+1D465}: 𝑥
	\item \href{https://www.compart.com/en/unicode/U+1D466}{U+1D466}: 𝑦
	\item \href{https://www.compart.com/en/unicode/U+1D467}{U+1D467}: 𝑧
	\item \href{https://www.compart.com/en/unicode/U+1D47}{U+1D47}: ᵇ
	\item \href{https://www.compart.com/en/unicode/U+1D48}{U+1D48}: ᵈ
	\item \href{https://www.compart.com/en/unicode/U+1D49}{U+1D49}: ᵉ
	\item \href{https://www.compart.com/en/unicode/U+1D49C}{U+1D49C}: 𝒜
	\item \href{https://www.compart.com/en/unicode/U+1D49E}{U+1D49E}: 𝒞
	\item \href{https://www.compart.com/en/unicode/U+1D49F}{U+1D49F}: 𝒟
	\item \href{https://www.compart.com/en/unicode/U+1D4A2}{U+1D4A2}: 𝒢
	\item \href{https://www.compart.com/en/unicode/U+1D4A5}{U+1D4A5}: 𝒥
	\item \href{https://www.compart.com/en/unicode/U+1D4A6}{U+1D4A6}: 𝒦
	\item \href{https://www.compart.com/en/unicode/U+1D4A9}{U+1D4A9}: 𝒩
	\item \href{https://www.compart.com/en/unicode/U+1D4AA}{U+1D4AA}: 𝒪
	\item \href{https://www.compart.com/en/unicode/U+1D4AB}{U+1D4AB}: 𝒫
	\item \href{https://www.compart.com/en/unicode/U+1D4AC}{U+1D4AC}: 𝒬
	\item \href{https://www.compart.com/en/unicode/U+1D4AE}{U+1D4AE}: 𝒮
	\item \href{https://www.compart.com/en/unicode/U+1D4AF}{U+1D4AF}: 𝒯
	\item \href{https://www.compart.com/en/unicode/U+1D4B0}{U+1D4B0}: 𝒰
	\item \href{https://www.compart.com/en/unicode/U+1D4B1}{U+1D4B1}: 𝒱
	\item \href{https://www.compart.com/en/unicode/U+1D4B2}{U+1D4B2}: 𝒲
	\item \href{https://www.compart.com/en/unicode/U+1D4B3}{U+1D4B3}: 𝒳
	\item \href{https://www.compart.com/en/unicode/U+1D4B4}{U+1D4B4}: 𝒴
	\item \href{https://www.compart.com/en/unicode/U+1D4B5}{U+1D4B5}: 𝒵
	\item \href{https://www.compart.com/en/unicode/U+1D4D}{U+1D4D}: ᵍ
	\item \href{https://www.compart.com/en/unicode/U+1D4D0}{U+1D4D0}: 𝓐
	\item \href{https://www.compart.com/en/unicode/U+1D4D1}{U+1D4D1}: 𝓑
	\item \href{https://www.compart.com/en/unicode/U+1D4D2}{U+1D4D2}: 𝓒
	\item \href{https://www.compart.com/en/unicode/U+1D4D3}{U+1D4D3}: 𝓓
	\item \href{https://www.compart.com/en/unicode/U+1D4D4}{U+1D4D4}: 𝓔
	\item \href{https://www.compart.com/en/unicode/U+1D4D5}{U+1D4D5}: 𝓕
	\item \href{https://www.compart.com/en/unicode/U+1D4D6}{U+1D4D6}: 𝓖
	\item \href{https://www.compart.com/en/unicode/U+1D4D7}{U+1D4D7}: 𝓗
	\item \href{https://www.compart.com/en/unicode/U+1D4D8}{U+1D4D8}: 𝓘
	\item \href{https://www.compart.com/en/unicode/U+1D4D9}{U+1D4D9}: 𝓙
	\item \href{https://www.compart.com/en/unicode/U+1D4DA}{U+1D4DA}: 𝓚
	\item \href{https://www.compart.com/en/unicode/U+1D4DB}{U+1D4DB}: 𝓛
	\item \href{https://www.compart.com/en/unicode/U+1D4DC}{U+1D4DC}: 𝓜
	\item \href{https://www.compart.com/en/unicode/U+1D4DD}{U+1D4DD}: 𝓝
	\item \href{https://www.compart.com/en/unicode/U+1D4DE}{U+1D4DE}: 𝓞
	\item \href{https://www.compart.com/en/unicode/U+1D4DF}{U+1D4DF}: 𝓟
	\item \href{https://www.compart.com/en/unicode/U+1D4E0}{U+1D4E0}: 𝓠
	\item \href{https://www.compart.com/en/unicode/U+1D4E1}{U+1D4E1}: 𝓡
	\item \href{https://www.compart.com/en/unicode/U+1D4E2}{U+1D4E2}: 𝓢
	\item \href{https://www.compart.com/en/unicode/U+1D4E3}{U+1D4E3}: 𝓣
	\item \href{https://www.compart.com/en/unicode/U+1D4E4}{U+1D4E4}: 𝓤
	\item \href{https://www.compart.com/en/unicode/U+1D4E5}{U+1D4E5}: 𝓥
	\item \href{https://www.compart.com/en/unicode/U+1D4E6}{U+1D4E6}: 𝓦
	\item \href{https://www.compart.com/en/unicode/U+1D4E7}{U+1D4E7}: 𝓧
	\item \href{https://www.compart.com/en/unicode/U+1D4E8}{U+1D4E8}: 𝓨
	\item \href{https://www.compart.com/en/unicode/U+1D4E9}{U+1D4E9}: 𝓩
	\item \href{https://www.compart.com/en/unicode/U+1D4F}{U+1D4F}: ᵏ
	\item \href{https://www.compart.com/en/unicode/U+1D50}{U+1D50}: ᵐ
	\item \href{https://www.compart.com/en/unicode/U+1D504}{U+1D504}: 𝔄
	\item \href{https://www.compart.com/en/unicode/U+1D505}{U+1D505}: 𝔅
	\item \href{https://www.compart.com/en/unicode/U+1D507}{U+1D507}: 𝔇
	\item \href{https://www.compart.com/en/unicode/U+1D508}{U+1D508}: 𝔈
	\item \href{https://www.compart.com/en/unicode/U+1D509}{U+1D509}: 𝔉
	\item \href{https://www.compart.com/en/unicode/U+1D50A}{U+1D50A}: 𝔊
	\item \href{https://www.compart.com/en/unicode/U+1D50D}{U+1D50D}: 𝔍
	\item \href{https://www.compart.com/en/unicode/U+1D50E}{U+1D50E}: 𝔎
	\item \href{https://www.compart.com/en/unicode/U+1D50F}{U+1D50F}: 𝔏
	\item \href{https://www.compart.com/en/unicode/U+1D510}{U+1D510}: 𝔐
	\item \href{https://www.compart.com/en/unicode/U+1D511}{U+1D511}: 𝔑
	\item \href{https://www.compart.com/en/unicode/U+1D512}{U+1D512}: 𝔒
	\item \href{https://www.compart.com/en/unicode/U+1D513}{U+1D513}: 𝔓
	\item \href{https://www.compart.com/en/unicode/U+1D514}{U+1D514}: 𝔔
	\item \href{https://www.compart.com/en/unicode/U+1D516}{U+1D516}: 𝔖
	\item \href{https://www.compart.com/en/unicode/U+1D517}{U+1D517}: 𝔗
	\item \href{https://www.compart.com/en/unicode/U+1D518}{U+1D518}: 𝔘
	\item \href{https://www.compart.com/en/unicode/U+1D519}{U+1D519}: 𝔙
	\item \href{https://www.compart.com/en/unicode/U+1D51A}{U+1D51A}: 𝔚
	\item \href{https://www.compart.com/en/unicode/U+1D51B}{U+1D51B}: 𝔛
	\item \href{https://www.compart.com/en/unicode/U+1D51C}{U+1D51C}: 𝔜
	\item \href{https://www.compart.com/en/unicode/U+1D51E}{U+1D51E}: 𝔞
	\item \href{https://www.compart.com/en/unicode/U+1D51F}{U+1D51F}: 𝔟
	\item \href{https://www.compart.com/en/unicode/U+1D52}{U+1D52}: ᵒ
	\item \href{https://www.compart.com/en/unicode/U+1D520}{U+1D520}: 𝔠
	\item \href{https://www.compart.com/en/unicode/U+1D521}{U+1D521}: 𝔡
	\item \href{https://www.compart.com/en/unicode/U+1D522}{U+1D522}: 𝔢
	\item \href{https://www.compart.com/en/unicode/U+1D523}{U+1D523}: 𝔣
	\item \href{https://www.compart.com/en/unicode/U+1D524}{U+1D524}: 𝔤
	\item \href{https://www.compart.com/en/unicode/U+1D525}{U+1D525}: 𝔥
	\item \href{https://www.compart.com/en/unicode/U+1D526}{U+1D526}: 𝔦
	\item \href{https://www.compart.com/en/unicode/U+1D527}{U+1D527}: 𝔧
	\item \href{https://www.compart.com/en/unicode/U+1D528}{U+1D528}: 𝔨
	\item \href{https://www.compart.com/en/unicode/U+1D529}{U+1D529}: 𝔩
	\item \href{https://www.compart.com/en/unicode/U+1D52A}{U+1D52A}: 𝔪
	\item \href{https://www.compart.com/en/unicode/U+1D52B}{U+1D52B}: 𝔫
	\item \href{https://www.compart.com/en/unicode/U+1D52C}{U+1D52C}: 𝔬
	\item \href{https://www.compart.com/en/unicode/U+1D52D}{U+1D52D}: 𝔭
	\item \href{https://www.compart.com/en/unicode/U+1D52E}{U+1D52E}: 𝔮
	\item \href{https://www.compart.com/en/unicode/U+1D52F}{U+1D52F}: 𝔯
	\item \href{https://www.compart.com/en/unicode/U+1D530}{U+1D530}: 𝔰
	\item \href{https://www.compart.com/en/unicode/U+1D531}{U+1D531}: 𝔱
	\item \href{https://www.compart.com/en/unicode/U+1D532}{U+1D532}: 𝔲
	\item \href{https://www.compart.com/en/unicode/U+1D533}{U+1D533}: 𝔳
	\item \href{https://www.compart.com/en/unicode/U+1D534}{U+1D534}: 𝔴
	\item \href{https://www.compart.com/en/unicode/U+1D535}{U+1D535}: 𝔵
	\item \href{https://www.compart.com/en/unicode/U+1D536}{U+1D536}: 𝔶
	\item \href{https://www.compart.com/en/unicode/U+1D537}{U+1D537}: 𝔷
	\item \href{https://www.compart.com/en/unicode/U+1D538}{U+1D538}: 𝔸
	\item \href{https://www.compart.com/en/unicode/U+1D539}{U+1D539}: 𝔹
	\item \href{https://www.compart.com/en/unicode/U+1D53B}{U+1D53B}: 𝔻
	\item \href{https://www.compart.com/en/unicode/U+1D53C}{U+1D53C}: 𝔼
	\item \href{https://www.compart.com/en/unicode/U+1D53D}{U+1D53D}: 𝔽
	\item \href{https://www.compart.com/en/unicode/U+1D53E}{U+1D53E}: 𝔾
	\item \href{https://www.compart.com/en/unicode/U+1D540}{U+1D540}: 𝕀
	\item \href{https://www.compart.com/en/unicode/U+1D541}{U+1D541}: 𝕁
	\item \href{https://www.compart.com/en/unicode/U+1D542}{U+1D542}: 𝕂
	\item \href{https://www.compart.com/en/unicode/U+1D543}{U+1D543}: 𝕃
	\item \href{https://www.compart.com/en/unicode/U+1D544}{U+1D544}: 𝕄
	\item \href{https://www.compart.com/en/unicode/U+1D546}{U+1D546}: 𝕆
	\item \href{https://www.compart.com/en/unicode/U+1D54A}{U+1D54A}: 𝕊
	\item \href{https://www.compart.com/en/unicode/U+1D54B}{U+1D54B}: 𝕋
	\item \href{https://www.compart.com/en/unicode/U+1D54C}{U+1D54C}: 𝕌
	\item \href{https://www.compart.com/en/unicode/U+1D54D}{U+1D54D}: 𝕍
	\item \href{https://www.compart.com/en/unicode/U+1D54E}{U+1D54E}: 𝕎
	\item \href{https://www.compart.com/en/unicode/U+1D54F}{U+1D54F}: 𝕏
	\item \href{https://www.compart.com/en/unicode/U+1D550}{U+1D550}: 𝕐
	\item \href{https://www.compart.com/en/unicode/U+1D552}{U+1D552}: 𝕒
	\item \href{https://www.compart.com/en/unicode/U+1D553}{U+1D553}: 𝕓
	\item \href{https://www.compart.com/en/unicode/U+1D554}{U+1D554}: 𝕔
	\item \href{https://www.compart.com/en/unicode/U+1D555}{U+1D555}: 𝕕
	\item \href{https://www.compart.com/en/unicode/U+1D556}{U+1D556}: 𝕖
	\item \href{https://www.compart.com/en/unicode/U+1D557}{U+1D557}: 𝕗
	\item \href{https://www.compart.com/en/unicode/U+1D558}{U+1D558}: 𝕘
	\item \href{https://www.compart.com/en/unicode/U+1D559}{U+1D559}: 𝕙
	\item \href{https://www.compart.com/en/unicode/U+1D55A}{U+1D55A}: 𝕚
	\item \href{https://www.compart.com/en/unicode/U+1D55B}{U+1D55B}: 𝕛
	\item \href{https://www.compart.com/en/unicode/U+1D55C}{U+1D55C}: 𝕜
	\item \href{https://www.compart.com/en/unicode/U+1D55D}{U+1D55D}: 𝕝
	\item \href{https://www.compart.com/en/unicode/U+1D55E}{U+1D55E}: 𝕞
	\item \href{https://www.compart.com/en/unicode/U+1D55F}{U+1D55F}: 𝕟
	\item \href{https://www.compart.com/en/unicode/U+1D56}{U+1D56}: ᵖ
	\item \href{https://www.compart.com/en/unicode/U+1D560}{U+1D560}: 𝕠
	\item \href{https://www.compart.com/en/unicode/U+1D561}{U+1D561}: 𝕡
	\item \href{https://www.compart.com/en/unicode/U+1D562}{U+1D562}: 𝕢
	\item \href{https://www.compart.com/en/unicode/U+1D563}{U+1D563}: 𝕣
	\item \href{https://www.compart.com/en/unicode/U+1D564}{U+1D564}: 𝕤
	\item \href{https://www.compart.com/en/unicode/U+1D565}{U+1D565}: 𝕥
	\item \href{https://www.compart.com/en/unicode/U+1D566}{U+1D566}: 𝕦
	\item \href{https://www.compart.com/en/unicode/U+1D567}{U+1D567}: 𝕧
	\item \href{https://www.compart.com/en/unicode/U+1D568}{U+1D568}: 𝕨
	\item \href{https://www.compart.com/en/unicode/U+1D569}{U+1D569}: 𝕩
	\item \href{https://www.compart.com/en/unicode/U+1D56A}{U+1D56A}: 𝕪
	\item \href{https://www.compart.com/en/unicode/U+1D56B}{U+1D56B}: 𝕫
	\item \href{https://www.compart.com/en/unicode/U+1D57}{U+1D57}: ᵗ
	\item \href{https://www.compart.com/en/unicode/U+1D58}{U+1D58}: ᵘ
	\item \href{https://www.compart.com/en/unicode/U+1D5B}{U+1D5B}: ᵛ
	\item \href{https://www.compart.com/en/unicode/U+1D62}{U+1D62}: ᵢ
	\item \href{https://www.compart.com/en/unicode/U+1D63}{U+1D63}: ᵣ
	\item \href{https://www.compart.com/en/unicode/U+1D64}{U+1D64}: ᵤ
	\item \href{https://www.compart.com/en/unicode/U+1D65}{U+1D65}: ᵥ
	\item \href{https://www.compart.com/en/unicode/U+1D6E4}{U+1D6E4}: 𝛤
	\item \href{https://www.compart.com/en/unicode/U+1D6E5}{U+1D6E5}: 𝛥
	\item \href{https://www.compart.com/en/unicode/U+1D6F1}{U+1D6F1}: 𝛱
	\item \href{https://www.compart.com/en/unicode/U+1D6F4}{U+1D6F4}: 𝛴
	\item \href{https://www.compart.com/en/unicode/U+1D6FA}{U+1D6FA}: 𝛺
	\item \href{https://www.compart.com/en/unicode/U+1D6FC}{U+1D6FC}: 𝛼
	\item \href{https://www.compart.com/en/unicode/U+1D6FD}{U+1D6FD}: 𝛽
	\item \href{https://www.compart.com/en/unicode/U+1D6FE}{U+1D6FE}: 𝛾
	\item \href{https://www.compart.com/en/unicode/U+1D6FF}{U+1D6FF}: 𝛿
	\item \href{https://www.compart.com/en/unicode/U+1D700}{U+1D700}: 𝜀
	\item \href{https://www.compart.com/en/unicode/U+1D701}{U+1D701}: 𝜁
	\item \href{https://www.compart.com/en/unicode/U+1D702}{U+1D702}: 𝜂
	\item \href{https://www.compart.com/en/unicode/U+1D703}{U+1D703}: 𝜃
	\item \href{https://www.compart.com/en/unicode/U+1D704}{U+1D704}: 𝜄
	\item \href{https://www.compart.com/en/unicode/U+1D705}{U+1D705}: 𝜅
	\item \href{https://www.compart.com/en/unicode/U+1D706}{U+1D706}: 𝜆
	\item \href{https://www.compart.com/en/unicode/U+1D707}{U+1D707}: 𝜇
	\item \href{https://www.compart.com/en/unicode/U+1D708}{U+1D708}: 𝜈
	\item \href{https://www.compart.com/en/unicode/U+1D709}{U+1D709}: 𝜉
	\item \href{https://www.compart.com/en/unicode/U+1D70A}{U+1D70A}: 𝜊
	\item \href{https://www.compart.com/en/unicode/U+1D70B}{U+1D70B}: 𝜋
	\item \href{https://www.compart.com/en/unicode/U+1D70C}{U+1D70C}: 𝜌
	\item \href{https://www.compart.com/en/unicode/U+1D70D}{U+1D70D}: 𝜍
	\item \href{https://www.compart.com/en/unicode/U+1D70E}{U+1D70E}: 𝜎
	\item \href{https://www.compart.com/en/unicode/U+1D70F}{U+1D70F}: 𝜏
	\item \href{https://www.compart.com/en/unicode/U+1D710}{U+1D710}: 𝜐
	\item \href{https://www.compart.com/en/unicode/U+1D711}{U+1D711}: 𝜑
	\item \href{https://www.compart.com/en/unicode/U+1D712}{U+1D712}: 𝜒
	\item \href{https://www.compart.com/en/unicode/U+1D713}{U+1D713}: 𝜓
	\item \href{https://www.compart.com/en/unicode/U+1D714}{U+1D714}: 𝜔
	\item \href{https://www.compart.com/en/unicode/U+1D716}{U+1D716}: 𝜖
	\item \href{https://www.compart.com/en/unicode/U+1D717}{U+1D717}: 𝜗
	\item \href{https://www.compart.com/en/unicode/U+1D719}{U+1D719}: 𝜙
	\item \href{https://www.compart.com/en/unicode/U+1D71A}{U+1D71A}: 𝜚
	\item \href{https://www.compart.com/en/unicode/U+1D71B}{U+1D71B}: 𝜛
	\item \href{https://www.compart.com/en/unicode/U+1D7D8}{U+1D7D8}: 𝟘
	\item \href{https://www.compart.com/en/unicode/U+1D7D9}{U+1D7D9}: 𝟙
	\item \href{https://www.compart.com/en/unicode/U+1D7DA}{U+1D7DA}: 𝟚
	\item \href{https://www.compart.com/en/unicode/U+1D7DB}{U+1D7DB}: 𝟛
	\item \href{https://www.compart.com/en/unicode/U+1D7DC}{U+1D7DC}: 𝟜
	\item \href{https://www.compart.com/en/unicode/U+1D7DD}{U+1D7DD}: 𝟝
	\item \href{https://www.compart.com/en/unicode/U+1D7DE}{U+1D7DE}: 𝟞
	\item \href{https://www.compart.com/en/unicode/U+1D7DF}{U+1D7DF}: 𝟟
	\item \href{https://www.compart.com/en/unicode/U+1D7E0}{U+1D7E0}: 𝟠
	\item \href{https://www.compart.com/en/unicode/U+1D7E1}{U+1D7E1}: 𝟡
	\item \href{https://www.compart.com/en/unicode/U+1D9C}{U+1D9C}: ᶜ
	\item \href{https://www.compart.com/en/unicode/U+1DA0}{U+1DA0}: ᶠ
	\item \href{https://www.compart.com/en/unicode/U+1DBB}{U+1DBB}: ᶻ
	\item \href{https://www.compart.com/en/unicode/U+1F329}{U+1F329}: 🌩
	\item \href{https://www.compart.com/en/unicode/U+2013}{U+2013}: –
	\item \href{https://www.compart.com/en/unicode/U+2014}{U+2014}: —
	\item \href{https://www.compart.com/en/unicode/U+2016}{U+2016}: ‖
	\item \href{https://www.compart.com/en/unicode/U+2018}{U+2018}: ‘
	\item \href{https://www.compart.com/en/unicode/U+2019}{U+2019}: ’
	\item \href{https://www.compart.com/en/unicode/U+201A}{U+201A}: ‚
	\item \href{https://www.compart.com/en/unicode/U+201C}{U+201C}: “
	\item \href{https://www.compart.com/en/unicode/U+201D}{U+201D}: ”
	\item \href{https://www.compart.com/en/unicode/U+201E}{U+201E}: „
	\item \href{https://www.compart.com/en/unicode/U+2020}{U+2020}: †
	\item \href{https://www.compart.com/en/unicode/U+2021}{U+2021}: ‡
	\item \href{https://www.compart.com/en/unicode/U+2022}{U+2022}: •
	\item \href{https://www.compart.com/en/unicode/U+2023}{U+2023}: ‣
	\item \href{https://www.compart.com/en/unicode/U+2026}{U+2026}: …
	\item \href{https://www.compart.com/en/unicode/U+202F}{U+202F}:  
	\item \href{https://www.compart.com/en/unicode/U+2030}{U+2030}: ‰
	\item \href{https://www.compart.com/en/unicode/U+2031}{U+2031}: ‱
	\item \href{https://www.compart.com/en/unicode/U+2032}{U+2032}: ′
	\item \href{https://www.compart.com/en/unicode/U+2033}{U+2033}: ″
	\item \href{https://www.compart.com/en/unicode/U+2034}{U+2034}: ‴
	\item \href{https://www.compart.com/en/unicode/U+2035}{U+2035}: ‵
	\item \href{https://www.compart.com/en/unicode/U+2038}{U+2038}: ‸
	\item \href{https://www.compart.com/en/unicode/U+2039}{U+2039}: ‹
	\item \href{https://www.compart.com/en/unicode/U+203A}{U+203A}: ›
	\item \href{https://www.compart.com/en/unicode/U+203B}{U+203B}: ※
	\item \href{https://www.compart.com/en/unicode/U+203C}{U+203C}: ‼
	\item \href{https://www.compart.com/en/unicode/U+203D}{U+203D}: ‽
	\item \href{https://www.compart.com/en/unicode/U+203E}{U+203E}: ‾
	\item \href{https://www.compart.com/en/unicode/U+2042}{U+2042}: ⁂
	\item \href{https://www.compart.com/en/unicode/U+2045}{U+2045}: ⁅
	\item \href{https://www.compart.com/en/unicode/U+2046}{U+2046}: ⁆
	\item \href{https://www.compart.com/en/unicode/U+2047}{U+2047}: ⁇
	\item \href{https://www.compart.com/en/unicode/U+2048}{U+2048}: ⁈
	\item \href{https://www.compart.com/en/unicode/U+2049}{U+2049}: ⁉
	\item \href{https://www.compart.com/en/unicode/U+2052}{U+2052}: ⁒
	\item \href{https://www.compart.com/en/unicode/U+2062}{U+2062}: ⁢
	\item \href{https://www.compart.com/en/unicode/U+2070}{U+2070}: ⁰
	\item \href{https://www.compart.com/en/unicode/U+2071}{U+2071}: ⁱ
	\item \href{https://www.compart.com/en/unicode/U+2074}{U+2074}: ⁴
	\item \href{https://www.compart.com/en/unicode/U+2075}{U+2075}: ⁵
	\item \href{https://www.compart.com/en/unicode/U+2076}{U+2076}: ⁶
	\item \href{https://www.compart.com/en/unicode/U+2077}{U+2077}: ⁷
	\item \href{https://www.compart.com/en/unicode/U+2078}{U+2078}: ⁸
	\item \href{https://www.compart.com/en/unicode/U+2079}{U+2079}: ⁹
	\item \href{https://www.compart.com/en/unicode/U+207A}{U+207A}: ⁺
	\item \href{https://www.compart.com/en/unicode/U+207B}{U+207B}: ⁻
	\item \href{https://www.compart.com/en/unicode/U+207C}{U+207C}: ⁼
	\item \href{https://www.compart.com/en/unicode/U+207D}{U+207D}: ⁽
	\item \href{https://www.compart.com/en/unicode/U+207E}{U+207E}: ⁾
	\item \href{https://www.compart.com/en/unicode/U+207F}{U+207F}: ⁿ
	\item \href{https://www.compart.com/en/unicode/U+2080}{U+2080}: ₀
	\item \href{https://www.compart.com/en/unicode/U+2081}{U+2081}: ₁
	\item \href{https://www.compart.com/en/unicode/U+2082}{U+2082}: ₂
	\item \href{https://www.compart.com/en/unicode/U+2083}{U+2083}: ₃
	\item \href{https://www.compart.com/en/unicode/U+2084}{U+2084}: ₄
	\item \href{https://www.compart.com/en/unicode/U+2085}{U+2085}: ₅
	\item \href{https://www.compart.com/en/unicode/U+2086}{U+2086}: ₆
	\item \href{https://www.compart.com/en/unicode/U+2087}{U+2087}: ₇
	\item \href{https://www.compart.com/en/unicode/U+2088}{U+2088}: ₈
	\item \href{https://www.compart.com/en/unicode/U+2089}{U+2089}: ₉
	\item \href{https://www.compart.com/en/unicode/U+208A}{U+208A}: ₊
	\item \href{https://www.compart.com/en/unicode/U+208B}{U+208B}: ₋
	\item \href{https://www.compart.com/en/unicode/U+208C}{U+208C}: ₌
	\item \href{https://www.compart.com/en/unicode/U+208D}{U+208D}: ₍
	\item \href{https://www.compart.com/en/unicode/U+208E}{U+208E}: ₎
	\item \href{https://www.compart.com/en/unicode/U+2090}{U+2090}: ₐ
	\item \href{https://www.compart.com/en/unicode/U+2091}{U+2091}: ₑ
	\item \href{https://www.compart.com/en/unicode/U+2092}{U+2092}: ₒ
	\item \href{https://www.compart.com/en/unicode/U+2093}{U+2093}: ₓ
	\item \href{https://www.compart.com/en/unicode/U+2095}{U+2095}: ₕ
	\item \href{https://www.compart.com/en/unicode/U+2096}{U+2096}: ₖ
	\item \href{https://www.compart.com/en/unicode/U+2097}{U+2097}: ₗ
	\item \href{https://www.compart.com/en/unicode/U+2098}{U+2098}: ₘ
	\item \href{https://www.compart.com/en/unicode/U+2099}{U+2099}: ₙ
	\item \href{https://www.compart.com/en/unicode/U+209A}{U+209A}: ₚ
	\item \href{https://www.compart.com/en/unicode/U+209B}{U+209B}: ₛ
	\item \href{https://www.compart.com/en/unicode/U+209C}{U+209C}: ₜ
	\item \href{https://www.compart.com/en/unicode/U+20AC}{U+20AC}: €
	\item \href{https://www.compart.com/en/unicode/U+2102}{U+2102}: ℂ
	\item \href{https://www.compart.com/en/unicode/U+2107}{U+2107}: ℇ
	\item \href{https://www.compart.com/en/unicode/U+210A}{U+210A}: ℊ
	\item \href{https://www.compart.com/en/unicode/U+210B}{U+210B}: ℋ
	\item \href{https://www.compart.com/en/unicode/U+210C}{U+210C}: ℌ
	\item \href{https://www.compart.com/en/unicode/U+210D}{U+210D}: ℍ
	\item \href{https://www.compart.com/en/unicode/U+210E}{U+210E}: ℎ
	\item \href{https://www.compart.com/en/unicode/U+210F}{U+210F}: ℏ
	\item \href{https://www.compart.com/en/unicode/U+2110}{U+2110}: ℐ
	\item \href{https://www.compart.com/en/unicode/U+2111}{U+2111}: ℑ
	\item \href{https://www.compart.com/en/unicode/U+2112}{U+2112}: ℒ
	\item \href{https://www.compart.com/en/unicode/U+2113}{U+2113}: ℓ
	\item \href{https://www.compart.com/en/unicode/U+2115}{U+2115}: ℕ
	\item \href{https://www.compart.com/en/unicode/U+2118}{U+2118}: ℘
	\item \href{https://www.compart.com/en/unicode/U+2119}{U+2119}: ℙ
	\item \href{https://www.compart.com/en/unicode/U+211A}{U+211A}: ℚ
	\item \href{https://www.compart.com/en/unicode/U+211B}{U+211B}: ℛ
	\item \href{https://www.compart.com/en/unicode/U+211C}{U+211C}: ℜ
	\item \href{https://www.compart.com/en/unicode/U+211D}{U+211D}: ℝ
	\item \href{https://www.compart.com/en/unicode/U+2122}{U+2122}: ™
	\item \href{https://www.compart.com/en/unicode/U+2124}{U+2124}: ℤ
	\item \href{https://www.compart.com/en/unicode/U+2126}{U+2126}: Ω
	\item \href{https://www.compart.com/en/unicode/U+2127}{U+2127}: ℧
	\item \href{https://www.compart.com/en/unicode/U+2128}{U+2128}: ℨ
	\item \href{https://www.compart.com/en/unicode/U+212A}{U+212A}: K
	\item \href{https://www.compart.com/en/unicode/U+212B}{U+212B}: Å
	\item \href{https://www.compart.com/en/unicode/U+212C}{U+212C}: ℬ
	\item \href{https://www.compart.com/en/unicode/U+212D}{U+212D}: ℭ
	\item \href{https://www.compart.com/en/unicode/U+212E}{U+212E}: ℮
	\item \href{https://www.compart.com/en/unicode/U+212F}{U+212F}: ℯ
	\item \href{https://www.compart.com/en/unicode/U+2130}{U+2130}: ℰ
	\item \href{https://www.compart.com/en/unicode/U+2131}{U+2131}: ℱ
	\item \href{https://www.compart.com/en/unicode/U+2132}{U+2132}: Ⅎ
	\item \href{https://www.compart.com/en/unicode/U+2133}{U+2133}: ℳ
	\item \href{https://www.compart.com/en/unicode/U+2135}{U+2135}: ℵ
	\item \href{https://www.compart.com/en/unicode/U+2136}{U+2136}: ℶ
	\item \href{https://www.compart.com/en/unicode/U+2137}{U+2137}: ℷ
	\item \href{https://www.compart.com/en/unicode/U+2138}{U+2138}: ℸ
	\item \href{https://www.compart.com/en/unicode/U+213C}{U+213C}: ℼ
	\item \href{https://www.compart.com/en/unicode/U+213D}{U+213D}: ℽ
	\item \href{https://www.compart.com/en/unicode/U+213E}{U+213E}: ℾ
	\item \href{https://www.compart.com/en/unicode/U+213F}{U+213F}: ℿ
	\item \href{https://www.compart.com/en/unicode/U+2140}{U+2140}: ⅀
	\item \href{https://www.compart.com/en/unicode/U+2141}{U+2141}: ⅁
	\item \href{https://www.compart.com/en/unicode/U+2144}{U+2144}: ⅄
	\item \href{https://www.compart.com/en/unicode/U+2146}{U+2146}: ⅆ
	\item \href{https://www.compart.com/en/unicode/U+2148}{U+2148}: ⅈ
	\item \href{https://www.compart.com/en/unicode/U+2149}{U+2149}: ⅉ
	\item \href{https://www.compart.com/en/unicode/U+214B}{U+214B}: ⅋
	\item \href{https://www.compart.com/en/unicode/U+2153}{U+2153}: ⅓
	\item \href{https://www.compart.com/en/unicode/U+2154}{U+2154}: ⅔
	\item \href{https://www.compart.com/en/unicode/U+2155}{U+2155}: ⅕
	\item \href{https://www.compart.com/en/unicode/U+2156}{U+2156}: ⅖
	\item \href{https://www.compart.com/en/unicode/U+2157}{U+2157}: ⅗
	\item \href{https://www.compart.com/en/unicode/U+2158}{U+2158}: ⅘
	\item \href{https://www.compart.com/en/unicode/U+2159}{U+2159}: ⅙
	\item \href{https://www.compart.com/en/unicode/U+215A}{U+215A}: ⅚
	\item \href{https://www.compart.com/en/unicode/U+215B}{U+215B}: ⅛
	\item \href{https://www.compart.com/en/unicode/U+215D}{U+215D}: ⅝
	\item \href{https://www.compart.com/en/unicode/U+215E}{U+215E}: ⅞
	\item \href{https://www.compart.com/en/unicode/U+2190}{U+2190}: ←
	\item \href{https://www.compart.com/en/unicode/U+2191}{U+2191}: ↑
	\item \href{https://www.compart.com/en/unicode/U+2192}{U+2192}: →
	\item \href{https://www.compart.com/en/unicode/U+2193}{U+2193}: ↓
	\item \href{https://www.compart.com/en/unicode/U+2194}{U+2194}: ↔
	\item \href{https://www.compart.com/en/unicode/U+2195}{U+2195}: ↕
	\item \href{https://www.compart.com/en/unicode/U+2196}{U+2196}: ↖
	\item \href{https://www.compart.com/en/unicode/U+2197}{U+2197}: ↗
	\item \href{https://www.compart.com/en/unicode/U+2198}{U+2198}: ↘
	\item \href{https://www.compart.com/en/unicode/U+2199}{U+2199}: ↙
	\item \href{https://www.compart.com/en/unicode/U+219A}{U+219A}: ↚
	\item \href{https://www.compart.com/en/unicode/U+219B}{U+219B}: ↛
	\item \href{https://www.compart.com/en/unicode/U+219E}{U+219E}: ↞
	\item \href{https://www.compart.com/en/unicode/U+21A0}{U+21A0}: ↠
	\item \href{https://www.compart.com/en/unicode/U+21A2}{U+21A2}: ↢
	\item \href{https://www.compart.com/en/unicode/U+21A3}{U+21A3}: ↣
	\item \href{https://www.compart.com/en/unicode/U+21A4}{U+21A4}: ↤
	\item \href{https://www.compart.com/en/unicode/U+21A6}{U+21A6}: ↦
	\item \href{https://www.compart.com/en/unicode/U+21A9}{U+21A9}: ↩
	\item \href{https://www.compart.com/en/unicode/U+21AA}{U+21AA}: ↪
	\item \href{https://www.compart.com/en/unicode/U+21AB}{U+21AB}: ↫
	\item \href{https://www.compart.com/en/unicode/U+21AC}{U+21AC}: ↬
	\item \href{https://www.compart.com/en/unicode/U+21AD}{U+21AD}: ↭
	\item \href{https://www.compart.com/en/unicode/U+21AE}{U+21AE}: ↮
	\item \href{https://www.compart.com/en/unicode/U+21AF}{U+21AF}: ↯
	\item \href{https://www.compart.com/en/unicode/U+21B0}{U+21B0}: ↰
	\item \href{https://www.compart.com/en/unicode/U+21B1}{U+21B1}: ↱
	\item \href{https://www.compart.com/en/unicode/U+21B6}{U+21B6}: ↶
	\item \href{https://www.compart.com/en/unicode/U+21B7}{U+21B7}: ↷
	\item \href{https://www.compart.com/en/unicode/U+21BA}{U+21BA}: ↺
	\item \href{https://www.compart.com/en/unicode/U+21BB}{U+21BB}: ↻
	\item \href{https://www.compart.com/en/unicode/U+21BC}{U+21BC}: ↼
	\item \href{https://www.compart.com/en/unicode/U+21BD}{U+21BD}: ↽
	\item \href{https://www.compart.com/en/unicode/U+21BE}{U+21BE}: ↾
	\item \href{https://www.compart.com/en/unicode/U+21BF}{U+21BF}: ↿
	\item \href{https://www.compart.com/en/unicode/U+21C0}{U+21C0}: ⇀
	\item \href{https://www.compart.com/en/unicode/U+21C1}{U+21C1}: ⇁
	\item \href{https://www.compart.com/en/unicode/U+21C2}{U+21C2}: ⇂
	\item \href{https://www.compart.com/en/unicode/U+21C3}{U+21C3}: ⇃
	\item \href{https://www.compart.com/en/unicode/U+21C4}{U+21C4}: ⇄
	\item \href{https://www.compart.com/en/unicode/U+21C5}{U+21C5}: ⇅
	\item \href{https://www.compart.com/en/unicode/U+21C6}{U+21C6}: ⇆
	\item \href{https://www.compart.com/en/unicode/U+21C7}{U+21C7}: ⇇
	\item \href{https://www.compart.com/en/unicode/U+21C8}{U+21C8}: ⇈
	\item \href{https://www.compart.com/en/unicode/U+21C9}{U+21C9}: ⇉
	\item \href{https://www.compart.com/en/unicode/U+21CA}{U+21CA}: ⇊
	\item \href{https://www.compart.com/en/unicode/U+21CB}{U+21CB}: ⇋
	\item \href{https://www.compart.com/en/unicode/U+21CC}{U+21CC}: ⇌
	\item \href{https://www.compart.com/en/unicode/U+21CD}{U+21CD}: ⇍
	\item \href{https://www.compart.com/en/unicode/U+21CE}{U+21CE}: ⇎
	\item \href{https://www.compart.com/en/unicode/U+21CF}{U+21CF}: ⇏
	\item \href{https://www.compart.com/en/unicode/U+21D0}{U+21D0}: ⇐
	\item \href{https://www.compart.com/en/unicode/U+21D1}{U+21D1}: ⇑
	\item \href{https://www.compart.com/en/unicode/U+21D2}{U+21D2}: ⇒
	\item \href{https://www.compart.com/en/unicode/U+21D3}{U+21D3}: ⇓
	\item \href{https://www.compart.com/en/unicode/U+21D4}{U+21D4}: ⇔
	\item \href{https://www.compart.com/en/unicode/U+21D5}{U+21D5}: ⇕
	\item \href{https://www.compart.com/en/unicode/U+21D6}{U+21D6}: ⇖
	\item \href{https://www.compart.com/en/unicode/U+21D7}{U+21D7}: ⇗
	\item \href{https://www.compart.com/en/unicode/U+21D8}{U+21D8}: ⇘
	\item \href{https://www.compart.com/en/unicode/U+21D9}{U+21D9}: ⇙
	\item \href{https://www.compart.com/en/unicode/U+21DA}{U+21DA}: ⇚
	\item \href{https://www.compart.com/en/unicode/U+21DB}{U+21DB}: ⇛
	\item \href{https://www.compart.com/en/unicode/U+21DC}{U+21DC}: ⇜
	\item \href{https://www.compart.com/en/unicode/U+21DD}{U+21DD}: ⇝
	\item \href{https://www.compart.com/en/unicode/U+21E0}{U+21E0}: ⇠
	\item \href{https://www.compart.com/en/unicode/U+21E2}{U+21E2}: ⇢
	\item \href{https://www.compart.com/en/unicode/U+21E4}{U+21E4}: ⇤
	\item \href{https://www.compart.com/en/unicode/U+21E5}{U+21E5}: ⇥
	\item \href{https://www.compart.com/en/unicode/U+21F0}{U+21F0}: ⇰
	\item \href{https://www.compart.com/en/unicode/U+21FD}{U+21FD}: ⇽
	\item \href{https://www.compart.com/en/unicode/U+21FE}{U+21FE}: ⇾
	\item \href{https://www.compart.com/en/unicode/U+21FF}{U+21FF}: ⇿
	\item \href{https://www.compart.com/en/unicode/U+2200}{U+2200}: ∀
	\item \href{https://www.compart.com/en/unicode/U+2201}{U+2201}: ∁
	\item \href{https://www.compart.com/en/unicode/U+2202}{U+2202}: ∂
	\item \href{https://www.compart.com/en/unicode/U+2203}{U+2203}: ∃
	\item \href{https://www.compart.com/en/unicode/U+2204}{U+2204}: ∄
	\item \href{https://www.compart.com/en/unicode/U+2205}{U+2205}: ∅
	\item \href{https://www.compart.com/en/unicode/U+2207}{U+2207}: ∇
	\item \href{https://www.compart.com/en/unicode/U+2208}{U+2208}: ∈
	\item \href{https://www.compart.com/en/unicode/U+2209}{U+2209}: ∉
	\item \href{https://www.compart.com/en/unicode/U+220B}{U+220B}: ∋
	\item \href{https://www.compart.com/en/unicode/U+220C}{U+220C}: ∌
	\item \href{https://www.compart.com/en/unicode/U+220D}{U+220D}: ∍
	\item \href{https://www.compart.com/en/unicode/U+220E}{U+220E}: ∎
	\item \href{https://www.compart.com/en/unicode/U+220F}{U+220F}: ∏
	\item \href{https://www.compart.com/en/unicode/U+2210}{U+2210}: ∐
	\item \href{https://www.compart.com/en/unicode/U+2211}{U+2211}: ∑
	\item \href{https://www.compart.com/en/unicode/U+2212}{U+2212}: −
	\item \href{https://www.compart.com/en/unicode/U+2213}{U+2213}: ∓
	\item \href{https://www.compart.com/en/unicode/U+2214}{U+2214}: ∔
	\item \href{https://www.compart.com/en/unicode/U+2215}{U+2215}: ∕
	\item \href{https://www.compart.com/en/unicode/U+2216}{U+2216}: ∖
	\item \href{https://www.compart.com/en/unicode/U+2217}{U+2217}: ∗
	\item \href{https://www.compart.com/en/unicode/U+2218}{U+2218}: ∘
	\item \href{https://www.compart.com/en/unicode/U+2219}{U+2219}: ∙
	\item \href{https://www.compart.com/en/unicode/U+221A}{U+221A}: √
	\item \href{https://www.compart.com/en/unicode/U+221B}{U+221B}: ∛
	\item \href{https://www.compart.com/en/unicode/U+221C}{U+221C}: ∜
	\item \href{https://www.compart.com/en/unicode/U+221D}{U+221D}: ∝
	\item \href{https://www.compart.com/en/unicode/U+221E}{U+221E}: ∞
	\item \href{https://www.compart.com/en/unicode/U+2220}{U+2220}: ∠
	\item \href{https://www.compart.com/en/unicode/U+2221}{U+2221}: ∡
	\item \href{https://www.compart.com/en/unicode/U+2222}{U+2222}: ∢
	\item \href{https://www.compart.com/en/unicode/U+2223}{U+2223}: ∣
	\item \href{https://www.compart.com/en/unicode/U+2224}{U+2224}: ∤
	\item \href{https://www.compart.com/en/unicode/U+2225}{U+2225}: ∥
	\item \href{https://www.compart.com/en/unicode/U+2226}{U+2226}: ∦
	\item \href{https://www.compart.com/en/unicode/U+2227}{U+2227}: ∧
	\item \href{https://www.compart.com/en/unicode/U+2228}{U+2228}: ∨
	\item \href{https://www.compart.com/en/unicode/U+2229}{U+2229}: ∩
	\item \href{https://www.compart.com/en/unicode/U+222A}{U+222A}: ∪
	\item \href{https://www.compart.com/en/unicode/U+222B}{U+222B}: ∫
	\item \href{https://www.compart.com/en/unicode/U+222C}{U+222C}: ∬
	\item \href{https://www.compart.com/en/unicode/U+222D}{U+222D}: ∭
	\item \href{https://www.compart.com/en/unicode/U+222E}{U+222E}: ∮
	\item \href{https://www.compart.com/en/unicode/U+222F}{U+222F}: ∯
	\item \href{https://www.compart.com/en/unicode/U+2230}{U+2230}: ∰
	\item \href{https://www.compart.com/en/unicode/U+2232}{U+2232}: ∲
	\item \href{https://www.compart.com/en/unicode/U+2233}{U+2233}: ∳
	\item \href{https://www.compart.com/en/unicode/U+2234}{U+2234}: ∴
	\item \href{https://www.compart.com/en/unicode/U+2235}{U+2235}: ∵
	\item \href{https://www.compart.com/en/unicode/U+2236}{U+2236}: ∶
	\item \href{https://www.compart.com/en/unicode/U+2237}{U+2237}: ∷
	\item \href{https://www.compart.com/en/unicode/U+2238}{U+2238}: ∸
	\item \href{https://www.compart.com/en/unicode/U+2239}{U+2239}: ∹
	\item \href{https://www.compart.com/en/unicode/U+223C}{U+223C}: ∼
	\item \href{https://www.compart.com/en/unicode/U+223D}{U+223D}: ∽
	\item \href{https://www.compart.com/en/unicode/U+223F}{U+223F}: ∿
	\item \href{https://www.compart.com/en/unicode/U+2240}{U+2240}: ≀
	\item \href{https://www.compart.com/en/unicode/U+2241}{U+2241}: ≁
	\item \href{https://www.compart.com/en/unicode/U+2243}{U+2243}: ≃
	\item \href{https://www.compart.com/en/unicode/U+2244}{U+2244}: ≄
	\item \href{https://www.compart.com/en/unicode/U+2245}{U+2245}: ≅
	\item \href{https://www.compart.com/en/unicode/U+2247}{U+2247}: ≇
	\item \href{https://www.compart.com/en/unicode/U+2248}{U+2248}: ≈
	\item \href{https://www.compart.com/en/unicode/U+2249}{U+2249}: ≉
	\item \href{https://www.compart.com/en/unicode/U+224A}{U+224A}: ≊
	\item \href{https://www.compart.com/en/unicode/U+224D}{U+224D}: ≍
	\item \href{https://www.compart.com/en/unicode/U+224E}{U+224E}: ≎
	\item \href{https://www.compart.com/en/unicode/U+224F}{U+224F}: ≏
	\item \href{https://www.compart.com/en/unicode/U+2250}{U+2250}: ≐
	\item \href{https://www.compart.com/en/unicode/U+2251}{U+2251}: ≑
	\item \href{https://www.compart.com/en/unicode/U+2252}{U+2252}: ≒
	\item \href{https://www.compart.com/en/unicode/U+2253}{U+2253}: ≓
	\item \href{https://www.compart.com/en/unicode/U+2254}{U+2254}: ≔
	\item \href{https://www.compart.com/en/unicode/U+2255}{U+2255}: ≕
	\item \href{https://www.compart.com/en/unicode/U+2256}{U+2256}: ≖
	\item \href{https://www.compart.com/en/unicode/U+2257}{U+2257}: ≗
	\item \href{https://www.compart.com/en/unicode/U+2258}{U+2258}: ≘
	\item \href{https://www.compart.com/en/unicode/U+2259}{U+2259}: ≙
	\item \href{https://www.compart.com/en/unicode/U+225A}{U+225A}: ≚
	\item \href{https://www.compart.com/en/unicode/U+225B}{U+225B}: ≛
	\item \href{https://www.compart.com/en/unicode/U+225C}{U+225C}: ≜
	\item \href{https://www.compart.com/en/unicode/U+225D}{U+225D}: ≝
	\item \href{https://www.compart.com/en/unicode/U+225F}{U+225F}: ≟
	\item \href{https://www.compart.com/en/unicode/U+2260}{U+2260}: ≠
	\item \href{https://www.compart.com/en/unicode/U+2261}{U+2261}: ≡
	\item \href{https://www.compart.com/en/unicode/U+2262}{U+2262}: ≢
	\item \href{https://www.compart.com/en/unicode/U+2263}{U+2263}: ≣
	\item \href{https://www.compart.com/en/unicode/U+2264}{U+2264}: ≤
	\item \href{https://www.compart.com/en/unicode/U+2265}{U+2265}: ≥
	\item \href{https://www.compart.com/en/unicode/U+2266}{U+2266}: ≦
	\item \href{https://www.compart.com/en/unicode/U+2267}{U+2267}: ≧
	\item \href{https://www.compart.com/en/unicode/U+2268}{U+2268}: ≨
	\item \href{https://www.compart.com/en/unicode/U+2269}{U+2269}: ≩
	\item \href{https://www.compart.com/en/unicode/U+226A}{U+226A}: ≪
	\item \href{https://www.compart.com/en/unicode/U+226B}{U+226B}: ≫
	\item \href{https://www.compart.com/en/unicode/U+226C}{U+226C}: ≬
	\item \href{https://www.compart.com/en/unicode/U+226D}{U+226D}: ≭
	\item \href{https://www.compart.com/en/unicode/U+226E}{U+226E}: ≮
	\item \href{https://www.compart.com/en/unicode/U+226F}{U+226F}: ≯
	\item \href{https://www.compart.com/en/unicode/U+2270}{U+2270}: ≰
	\item \href{https://www.compart.com/en/unicode/U+2271}{U+2271}: ≱
	\item \href{https://www.compart.com/en/unicode/U+2272}{U+2272}: ≲
	\item \href{https://www.compart.com/en/unicode/U+2273}{U+2273}: ≳
	\item \href{https://www.compart.com/en/unicode/U+2274}{U+2274}: ≴
	\item \href{https://www.compart.com/en/unicode/U+2275}{U+2275}: ≵
	\item \href{https://www.compart.com/en/unicode/U+2276}{U+2276}: ≶
	\item \href{https://www.compart.com/en/unicode/U+2277}{U+2277}: ≷
	\item \href{https://www.compart.com/en/unicode/U+2278}{U+2278}: ≸
	\item \href{https://www.compart.com/en/unicode/U+2279}{U+2279}: ≹
	\item \href{https://www.compart.com/en/unicode/U+227A}{U+227A}: ≺
	\item \href{https://www.compart.com/en/unicode/U+227B}{U+227B}: ≻
	\item \href{https://www.compart.com/en/unicode/U+227C}{U+227C}: ≼
	\item \href{https://www.compart.com/en/unicode/U+227D}{U+227D}: ≽
	\item \href{https://www.compart.com/en/unicode/U+227E}{U+227E}: ≾
	\item \href{https://www.compart.com/en/unicode/U+227F}{U+227F}: ≿
	\item \href{https://www.compart.com/en/unicode/U+2280}{U+2280}: ⊀
	\item \href{https://www.compart.com/en/unicode/U+2281}{U+2281}: ⊁
	\item \href{https://www.compart.com/en/unicode/U+2282}{U+2282}: ⊂
	\item \href{https://www.compart.com/en/unicode/U+2283}{U+2283}: ⊃
	\item \href{https://www.compart.com/en/unicode/U+2284}{U+2284}: ⊄
	\item \href{https://www.compart.com/en/unicode/U+2285}{U+2285}: ⊅
	\item \href{https://www.compart.com/en/unicode/U+2286}{U+2286}: ⊆
	\item \href{https://www.compart.com/en/unicode/U+2287}{U+2287}: ⊇
	\item \href{https://www.compart.com/en/unicode/U+2288}{U+2288}: ⊈
	\item \href{https://www.compart.com/en/unicode/U+2289}{U+2289}: ⊉
	\item \href{https://www.compart.com/en/unicode/U+228A}{U+228A}: ⊊
	\item \href{https://www.compart.com/en/unicode/U+228B}{U+228B}: ⊋
	\item \href{https://www.compart.com/en/unicode/U+228E}{U+228E}: ⊎
	\item \href{https://www.compart.com/en/unicode/U+228F}{U+228F}: ⊏
	\item \href{https://www.compart.com/en/unicode/U+2290}{U+2290}: ⊐
	\item \href{https://www.compart.com/en/unicode/U+2291}{U+2291}: ⊑
	\item \href{https://www.compart.com/en/unicode/U+2292}{U+2292}: ⊒
	\item \href{https://www.compart.com/en/unicode/U+2293}{U+2293}: ⊓
	\item \href{https://www.compart.com/en/unicode/U+2294}{U+2294}: ⊔
	\item \href{https://www.compart.com/en/unicode/U+2295}{U+2295}: ⊕
	\item \href{https://www.compart.com/en/unicode/U+2296}{U+2296}: ⊖
	\item \href{https://www.compart.com/en/unicode/U+2297}{U+2297}: ⊗
	\item \href{https://www.compart.com/en/unicode/U+2298}{U+2298}: ⊘
	\item \href{https://www.compart.com/en/unicode/U+2299}{U+2299}: ⊙
	\item \href{https://www.compart.com/en/unicode/U+229A}{U+229A}: ⊚
	\item \href{https://www.compart.com/en/unicode/U+229B}{U+229B}: ⊛
	\item \href{https://www.compart.com/en/unicode/U+229D}{U+229D}: ⊝
	\item \href{https://www.compart.com/en/unicode/U+229E}{U+229E}: ⊞
	\item \href{https://www.compart.com/en/unicode/U+229F}{U+229F}: ⊟
	\item \href{https://www.compart.com/en/unicode/U+22A0}{U+22A0}: ⊠
	\item \href{https://www.compart.com/en/unicode/U+22A1}{U+22A1}: ⊡
	\item \href{https://www.compart.com/en/unicode/U+22A2}{U+22A2}: ⊢
	\item \href{https://www.compart.com/en/unicode/U+22A3}{U+22A3}: ⊣
	\item \href{https://www.compart.com/en/unicode/U+22A4}{U+22A4}: ⊤
	\item \href{https://www.compart.com/en/unicode/U+22A5}{U+22A5}: ⊥
	\item \href{https://www.compart.com/en/unicode/U+22A6}{U+22A6}: ⊦
	\item \href{https://www.compart.com/en/unicode/U+22A7}{U+22A7}: ⊧
	\item \href{https://www.compart.com/en/unicode/U+22A9}{U+22A9}: ⊩
	\item \href{https://www.compart.com/en/unicode/U+22AA}{U+22AA}: ⊪
	\item \href{https://www.compart.com/en/unicode/U+22AB}{U+22AB}: ⊫
	\item \href{https://www.compart.com/en/unicode/U+22AC}{U+22AC}: ⊬
	\item \href{https://www.compart.com/en/unicode/U+22AD}{U+22AD}: ⊭
	\item \href{https://www.compart.com/en/unicode/U+22AE}{U+22AE}: ⊮
	\item \href{https://www.compart.com/en/unicode/U+22AF}{U+22AF}: ⊯
	\item \href{https://www.compart.com/en/unicode/U+22B2}{U+22B2}: ⊲
	\item \href{https://www.compart.com/en/unicode/U+22B3}{U+22B3}: ⊳
	\item \href{https://www.compart.com/en/unicode/U+22B4}{U+22B4}: ⊴
	\item \href{https://www.compart.com/en/unicode/U+22B5}{U+22B5}: ⊵
	\item \href{https://www.compart.com/en/unicode/U+22B8}{U+22B8}: ⊸
	\item \href{https://www.compart.com/en/unicode/U+22BA}{U+22BA}: ⊺
	\item \href{https://www.compart.com/en/unicode/U+22BB}{U+22BB}: ⊻
	\item \href{https://www.compart.com/en/unicode/U+22BC}{U+22BC}: ⊼
	\item \href{https://www.compart.com/en/unicode/U+22C0}{U+22C0}: ⋀
	\item \href{https://www.compart.com/en/unicode/U+22C1}{U+22C1}: ⋁
	\item \href{https://www.compart.com/en/unicode/U+22C2}{U+22C2}: ⋂
	\item \href{https://www.compart.com/en/unicode/U+22C3}{U+22C3}: ⋃
	\item \href{https://www.compart.com/en/unicode/U+22C4}{U+22C4}: ⋄
	\item \href{https://www.compart.com/en/unicode/U+22C5}{U+22C5}: ⋅
	\item \href{https://www.compart.com/en/unicode/U+22C6}{U+22C6}: ⋆
	\item \href{https://www.compart.com/en/unicode/U+22C7}{U+22C7}: ⋇
	\item \href{https://www.compart.com/en/unicode/U+22C8}{U+22C8}: ⋈
	\item \href{https://www.compart.com/en/unicode/U+22C9}{U+22C9}: ⋉
	\item \href{https://www.compart.com/en/unicode/U+22CA}{U+22CA}: ⋊
	\item \href{https://www.compart.com/en/unicode/U+22CB}{U+22CB}: ⋋
	\item \href{https://www.compart.com/en/unicode/U+22CC}{U+22CC}: ⋌
	\item \href{https://www.compart.com/en/unicode/U+22CD}{U+22CD}: ⋍
	\item \href{https://www.compart.com/en/unicode/U+22CE}{U+22CE}: ⋎
	\item \href{https://www.compart.com/en/unicode/U+22CF}{U+22CF}: ⋏
	\item \href{https://www.compart.com/en/unicode/U+22D0}{U+22D0}: ⋐
	\item \href{https://www.compart.com/en/unicode/U+22D1}{U+22D1}: ⋑
	\item \href{https://www.compart.com/en/unicode/U+22D2}{U+22D2}: ⋒
	\item \href{https://www.compart.com/en/unicode/U+22D3}{U+22D3}: ⋓
	\item \href{https://www.compart.com/en/unicode/U+22D4}{U+22D4}: ⋔
	\item \href{https://www.compart.com/en/unicode/U+22D6}{U+22D6}: ⋖
	\item \href{https://www.compart.com/en/unicode/U+22D7}{U+22D7}: ⋗
	\item \href{https://www.compart.com/en/unicode/U+22D8}{U+22D8}: ⋘
	\item \href{https://www.compart.com/en/unicode/U+22D9}{U+22D9}: ⋙
	\item \href{https://www.compart.com/en/unicode/U+22DA}{U+22DA}: ⋚
	\item \href{https://www.compart.com/en/unicode/U+22DB}{U+22DB}: ⋛
	\item \href{https://www.compart.com/en/unicode/U+22DE}{U+22DE}: ⋞
	\item \href{https://www.compart.com/en/unicode/U+22DF}{U+22DF}: ⋟
	\item \href{https://www.compart.com/en/unicode/U+22E0}{U+22E0}: ⋠
	\item \href{https://www.compart.com/en/unicode/U+22E1}{U+22E1}: ⋡
	\item \href{https://www.compart.com/en/unicode/U+22E2}{U+22E2}: ⋢
	\item \href{https://www.compart.com/en/unicode/U+22E3}{U+22E3}: ⋣
	\item \href{https://www.compart.com/en/unicode/U+22E4}{U+22E4}: ⋤
	\item \href{https://www.compart.com/en/unicode/U+22E5}{U+22E5}: ⋥
	\item \href{https://www.compart.com/en/unicode/U+22E6}{U+22E6}: ⋦
	\item \href{https://www.compart.com/en/unicode/U+22E7}{U+22E7}: ⋧
	\item \href{https://www.compart.com/en/unicode/U+22E8}{U+22E8}: ⋨
	\item \href{https://www.compart.com/en/unicode/U+22E9}{U+22E9}: ⋩
	\item \href{https://www.compart.com/en/unicode/U+22EA}{U+22EA}: ⋪
	\item \href{https://www.compart.com/en/unicode/U+22EB}{U+22EB}: ⋫
	\item \href{https://www.compart.com/en/unicode/U+22EC}{U+22EC}: ⋬
	\item \href{https://www.compart.com/en/unicode/U+22ED}{U+22ED}: ⋭
	\item \href{https://www.compart.com/en/unicode/U+22EE}{U+22EE}: ⋮
	\item \href{https://www.compart.com/en/unicode/U+22EF}{U+22EF}: ⋯
	\item \href{https://www.compart.com/en/unicode/U+22F0}{U+22F0}: ⋰
	\item \href{https://www.compart.com/en/unicode/U+22F1}{U+22F1}: ⋱
	\item \href{https://www.compart.com/en/unicode/U+2300}{U+2300}: ⌀
	\item \href{https://www.compart.com/en/unicode/U+2308}{U+2308}: ⌈
	\item \href{https://www.compart.com/en/unicode/U+2309}{U+2309}: ⌉
	\item \href{https://www.compart.com/en/unicode/U+230A}{U+230A}: ⌊
	\item \href{https://www.compart.com/en/unicode/U+230B}{U+230B}: ⌋
	\item \href{https://www.compart.com/en/unicode/U+2322}{U+2322}: ⌢
	\item \href{https://www.compart.com/en/unicode/U+2323}{U+2323}: ⌣
	\item \href{https://www.compart.com/en/unicode/U+2329}{U+2329}: 〈
	\item \href{https://www.compart.com/en/unicode/U+232A}{U+232A}: 〉
	\item \href{https://www.compart.com/en/unicode/U+23CE}{U+23CE}: ⏎
	\item \href{https://www.compart.com/en/unicode/U+2460}{U+2460}: ①
	\item \href{https://www.compart.com/en/unicode/U+2461}{U+2461}: ②
	\item \href{https://www.compart.com/en/unicode/U+2462}{U+2462}: ③
	\item \href{https://www.compart.com/en/unicode/U+2463}{U+2463}: ④
	\item \href{https://www.compart.com/en/unicode/U+2464}{U+2464}: ⑤
	\item \href{https://www.compart.com/en/unicode/U+2465}{U+2465}: ⑥
	\item \href{https://www.compart.com/en/unicode/U+2466}{U+2466}: ⑦
	\item \href{https://www.compart.com/en/unicode/U+2467}{U+2467}: ⑧
	\item \href{https://www.compart.com/en/unicode/U+2468}{U+2468}: ⑨
	\item \href{https://www.compart.com/en/unicode/U+25A1}{U+25A1}: □
	\item \href{https://www.compart.com/en/unicode/U+25B3}{U+25B3}: △
	\item \href{https://www.compart.com/en/unicode/U+25C5}{U+25C5}: ◅
	\item \href{https://www.compart.com/en/unicode/U+2610}{U+2610}: ☐
	\item \href{https://www.compart.com/en/unicode/U+2611}{U+2611}: ☑
	\item \href{https://www.compart.com/en/unicode/U+2615}{U+2615}: ☕
	\item \href{https://www.compart.com/en/unicode/U+2621}{U+2621}: ☡
	\item \href{https://www.compart.com/en/unicode/U+2627}{U+2627}: ☧
	\item \href{https://www.compart.com/en/unicode/U+2639}{U+2639}: ☹
	\item \href{https://www.compart.com/en/unicode/U+263A}{U+263A}: ☺
	\item \href{https://www.compart.com/en/unicode/U+2660}{U+2660}: ♠
	\item \href{https://www.compart.com/en/unicode/U+2661}{U+2661}: ♡
	\item \href{https://www.compart.com/en/unicode/U+2662}{U+2662}: ♢
	\item \href{https://www.compart.com/en/unicode/U+2663}{U+2663}: ♣
	\item \href{https://www.compart.com/en/unicode/U+266D}{U+266D}: ♭
	\item \href{https://www.compart.com/en/unicode/U+266E}{U+266E}: ♮
	\item \href{https://www.compart.com/en/unicode/U+266F}{U+266F}: ♯
	\item \href{https://www.compart.com/en/unicode/U+26A0}{U+26A0}: ⚠
	\item \href{https://www.compart.com/en/unicode/U+2713}{U+2713}: ✓
	\item \href{https://www.compart.com/en/unicode/U+27C2}{U+27C2}: ⟂
	\item \href{https://www.compart.com/en/unicode/U+27E6}{U+27E6}: ⟦
	\item \href{https://www.compart.com/en/unicode/U+27E7}{U+27E7}: ⟧
	\item \href{https://www.compart.com/en/unicode/U+27E8}{U+27E8}: ⟨
	\item \href{https://www.compart.com/en/unicode/U+27E9}{U+27E9}: ⟩
	\item \href{https://www.compart.com/en/unicode/U+27EA}{U+27EA}: ⟪
	\item \href{https://www.compart.com/en/unicode/U+27EB}{U+27EB}: ⟫
	\item \href{https://www.compart.com/en/unicode/U+27F5}{U+27F5}: ⟵
	\item \href{https://www.compart.com/en/unicode/U+27F6}{U+27F6}: ⟶
	\item \href{https://www.compart.com/en/unicode/U+2983}{U+2983}: ⦃
	\item \href{https://www.compart.com/en/unicode/U+2984}{U+2984}: ⦄
	\item \href{https://www.compart.com/en/unicode/U+2985}{U+2985}: ⦅
	\item \href{https://www.compart.com/en/unicode/U+2986}{U+2986}: ⦆
	\item \href{https://www.compart.com/en/unicode/U+2987}{U+2987}: ⦇
	\item \href{https://www.compart.com/en/unicode/U+2988}{U+2988}: ⦈
	\item \href{https://www.compart.com/en/unicode/U+29F5}{U+29F5}: ⧵
	\item \href{https://www.compart.com/en/unicode/U+2A00}{U+2A00}: ⨀
	\item \href{https://www.compart.com/en/unicode/U+2A01}{U+2A01}: ⨁
	\item \href{https://www.compart.com/en/unicode/U+2A02}{U+2A02}: ⨂
	\item \href{https://www.compart.com/en/unicode/U+2A05}{U+2A05}: ⨅
	\item \href{https://www.compart.com/en/unicode/U+2A06}{U+2A06}: ⨆
	\item \href{https://www.compart.com/en/unicode/U+2A0C}{U+2A0C}: ⨌
	\item \href{https://www.compart.com/en/unicode/U+2A1D}{U+2A1D}: ⨝
	\item \href{https://www.compart.com/en/unicode/U+2A3F}{U+2A3F}: ⨿
	\item \href{https://www.compart.com/en/unicode/U+2A7D}{U+2A7D}: ⩽
	\item \href{https://www.compart.com/en/unicode/U+2A7E}{U+2A7E}: ⩾
	\item \href{https://www.compart.com/en/unicode/U+2AA8}{U+2AA8}: ⪨
	\item \href{https://www.compart.com/en/unicode/U+2AA9}{U+2AA9}: ⪩
	\item \href{https://www.compart.com/en/unicode/U+2AAF}{U+2AAF}: ⪯
	\item \href{https://www.compart.com/en/unicode/U+2AB0}{U+2AB0}: ⪰
	\item \href{https://www.compart.com/en/unicode/U+2C7C}{U+2C7C}: ⱼ
	\item \href{https://www.compart.com/en/unicode/U+2E18}{U+2E18}: ⸘
	\item \href{https://www.compart.com/en/unicode/U+301A}{U+301A}: 〚
	\item \href{https://www.compart.com/en/unicode/U+301B}{U+301B}: 〛
	\item \href{https://www.compart.com/en/unicode/U+33D1}{U+33D1}: ㏑
	\item \href{https://www.compart.com/en/unicode/U+33D2}{U+33D2}: ㏒
	\item \href{https://www.compart.com/en/unicode/U+D7B0}{U+D7B0}: ힰ
\end{itemize}

\end{document}